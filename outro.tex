В результате работы достигнуты следующие результаты.
\begin{enumerate}
   \item Выполнен сравнительный анализ методов прогнозирования и представлена
      сравнительная таблица, в которую включены наиболее часто используемые методы
      прогнозирования: модели типа ARIMA и модели, составленные на базе нейронных сетей.
   \item Сформулированы рекомендации по выбору модели прогнозирования для управления
      автоматизированной системой, функционирующей в условиях нестационарной нагрузки.
   \item Предложен алгоритм аккумуляции обучающей выборки нейронной 
      сети. Разработанный алгоритм позволяет существенно увеличивать точность нейронной сети.
      С каждой итерацей алгоритма нейронная сеть обучается на тестовых данных, собираемых 
      во время работы сервиса, корректируются весовые коэффициенты нейронных связей.
   \item Разработан алгоритм распределения нестационарной нагрузки сервиса"=агрегатора
      между кешем и пересылкой запроса сторонним сервисам в зависимости от величины мгновенной
      нагрузки сервиса. 
   \item Выполнено сравнение исходной системы с системой, использующей разработанный 
      алгоритм. Среднее время ответа при низкой нагрузке уменьшилось на 1\%, а
      при высокой нагрузке и в условиях временных перегрузок системы изменение данной величины
      достигло 18\% в лучшую сторону.
\end{enumerate}
