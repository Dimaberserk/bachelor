{\bfseries Актуальность темы исследования.} 
В последнее время большое распространение получили различные 
сервисы-агрегаторы, объединяющие данные из нескольких источников
определённой тематики в один, что уменьшает затрачиваемое пользователем
время на поиск необходимой информации. 

Для обработки запроса от одного пользователя, агрегатору может 
потребоваться совершить десятки или даже сотни запросов к сторонним 
ресурсам. В силу того, что пользователи могут запрашивать одну и ту же 
информацию, самым распространённым способ снижения времени ответа 
является использование промежуточных буферов, так называемых кэшей. Но 
данный способ приводит к снижению актуальности предоставляемых 
агрегатором данных и может быть использован для предупреждения 
избыточной нагрузки системы, либо для хранения редко обновляемых данных, 
таких как словари. Словарь -- это набор статических данных используемых 
агрегатором для объединения информации получаемой с нескольких источников 
(например, для агрегатора отелей словарём будет являться список всех отелей 
по которым осуществляется поиск). 

Таким образом, разработка алгоритма распределения нестационарной нагрузки, 
определяющего источник данных агрегатора, является актуальной задачей. 
Однако для её решения необходим анализ или прогноз нагрузки системы.

{\bfseries Степень теоретической разработанности темы.}
В открытом доступе существует множество научных работ, описывающих 
методы прогнозирования временных рядов. Но материалы, описывающие 
проблему выбора метода прогнозирования для разработки алгоритма 
распределения нестационарной нагрузки, найти не удалось. Из сказанного выше 
можно сделать вывод о том, что рассматриваемая тема имеет низкую степень 
теоретической проработанности.

{\bfseries Объектом исследования} является прогнозирование нестационарной 
нагрузки сервиса"=агрегатора.

{\bfseries Предметом исследования} являются методы прогнозирования 
временных рядов.

{\bfseries Область исследования.} Проведённое исследование методов 
прогнозирования нагрузки вычислительной системы в пределах разработки 
алгоритма распределения запросов пользователей полностью соответствует 
специальности «Вычислительные машины, комплексы, системы и сети», а 
содержание выпускной квалификационной работы -- техническому
заданию.

{\bfseries Цель и задачи исследования.}  Целью работы является улучшение 
качества обслуживания пользователей сервиса-агрегатора за счет снижения 
среднего времени ответа.

Для достижения данной цели были поставлены следующие задачи:
\begin{enumerate}
	\item Выполнить сравнительный анализ методов прогнозирования.
	\item Разработать алгоритм распределения нестационарной нагрузки и его 
		программную реализацию на основе выбранного метода 
		прогнозирования.
	\item Выполнить сравнение среднего времени ответа исходной и 
		использующей разработанный алгоритм систем.
\end{enumerate}

{\bfseries Теоретическую основу исследования} составляют научные труды
отечественных и зарубежных авторов в области компьютерных технологий и
математической статистики.

{\bfseries Методологическую основу исследования} составляет эксперимент.

{\bfseries Научная новизна работы} заключается в следующем:
\begin{enumerate}
	\item В настоящее время не существует структурированных рекомендаций 
		по выбору метода прогнозирования при разработке алгоритмов 
		распределения нагрузки, представленных в данной работе. 
	\item Разработанный алгоритм распределения нестационарной нагрузки 
		может обеспечить улучшение качества обслуживания пользователей
		сервиса"=агрегатора.
\end{enumerate}

{\bfseries Практическая значимость} данной работы заключается в том, что 
разработанный алгоритм распределения нестационарной нагрузки будет
интегрирован в разрабатываемый сервис"=агрегатор. Кроме того, 
сформулированные рекомендации могут быть использованы для осуществления 
выбора метода прогнозирования при разработке собственного алгоритма 
распределения запросов пользователей.

{\bfseries Апробация результатов исследования.} Сформулированные 
рекомендации по выбору метода прогнозирования обсуждались на VII Конгрессе 
молодых учёных, а их сравнительный анализ был представлен на XLVII научной 
и учебно-методической конференции.

{\bfseries Объем и структура работы.} 
Выпускная квалификационная работа содержит 40 страниц машинописного 
текста, восемь рисунков, семь таблиц, девять формул и список литературы, 
включающий 25 источников. Структурно работа состоит из введения, трёх 
разделов и заключения. Во введении обоснована актуальность выбранной темы, 
а также новизна исследования. Определены цели и задачи, объект и предмет 
исследования. Представлена апробация результатов исследования. Первый 
раздел содержит обзор предметной области, рассматриваются широко 
используемые методы прогнозирования временных рядов. Проводится 
сравнительный анализ методов и формулируются рекомендации по выбору 
моделей прогнозирования. Второй раздел посвящён процессу разработки 
алгоритма распределения нестационарной нагрузки, в нём описывается 
составление требований для нового алгоритма, а также проблемы его 
разработки. Приведено описание разработанных алгоритмов аккумуляции 
обучающей выборки и распределения запросов пользователей. Третий раздел 
посвящен тестированию разработанного комплекса. Приводятся результаты 
тестирования под двумя типами нагрузок. В заключении сформулированы 
основные результаты работы.
