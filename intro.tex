{\bfseries Актуальность темы исследования.} 
В последнее время большое распространение получили 
сервисы-агрегаторы, объединяющие в единый массив данные из нескольких источников
заданной тематики для того, чтобы уменьшить затрачиваемое пользователем
время на поиск необходимой ему информации.
В традиционной модели предоставления данных, сервисы-агрегаторы не используются, 
а пользователи вынуждены самостоятельно посещать все ресурсы в поисках информации.
Сервис-агрегатор в автоматическом режиме и с заданным периодом копирует информацию из
множества источников и при получении пользовательского запроса
предоставляет актуальный срез данных сразу по всем источникам.

Для обработки запроса от одного пользователя, агрегатору может 
потребоваться совершить десятки, а иногда сотни запросов к сторонним 
ресурсам. В силу того, что пользователи могут выполнять одинаковые запросы,
самым распространённым способом повышения качества обслуживания,
за счёт снижения времени ответа,
является использование промежуточных буферов, так называемых кэшей. Но 
данный способ приводит к снижению актуальности предоставляемых 
агрегатором данных. Он может быть использован, например, для предупреждения 
избыточной нагрузки системы, либо для хранения редко обновляемых данных, 
таких как словари. Словарь -- это набор статических данных используемых 
агрегатором для объединения информации получаемой с нескольких источников 
(например, для агрегатора отелей словарём будет являться список всех отелей 
по которым осуществляется поиск). 

Очевидно, такой способ не подходит для часто меняющихся данных, поскольку
пользователи не получат актуального состояния в момент выполнения запроса.
Для решения этой задачи можно настроить сервис-агрегатор таким образом,
чтобы в зависимости от значения мгновенной нагрузки, автоматически
выбирался подходящий способ предоставления данных: при невысокой нагрузке -- 
актуальных данных, а при высокой -- данных из кеша.
В ходе выполнения обзора предметной области не было найдено описаний
алгоритмов для решения этой задачи, следовательно, 
разработка алгоритма распределения нестационарной нагрузки, 
определяющего источник данных агрегатора, является актуальной задачей. 
Однако для её решения необходим анализ и прогноз нагрузки системы.

{\bfseries Степень теоретической разработанности темы.}
В открытом доступе существует множество научных работ, описывающих 
методы прогнозирования временных рядов. Но материалы, описывающие 
проблему выбора метода прогнозирования для разработки алгоритма 
распределения нестационарной нагрузки, найти не удалось. 
Существующие работы, в своей массе, не описывают проблематику с точки зрения 
моделирования нагрузки, в то же время, есть работы, подробно рассматривающие
нестационарные процессы, но в ракурсе прогнозирования временных рядов,
а не функционирования вычислительных систем.
Таким образом можно сделать вывод о том, что рассматриваемая тема имеет низкую степень 
теоретической проработанности.

{\bfseries Объектом исследования} являются программные реализации информационных
сервисов-агрегаторов.

{\bfseries Предметом исследования} является анализ применимости методов
прогнозирования нестационарных процессов для решеия задачи распределения
нагрузке в сервисе-агрегаторе.

{\bfseries Область исследования.} Проведённое исследование методов 
прогнозирования нагрузки вычислительной системы в пределах разработки 
алгоритма распределения запросов пользователей полностью соответствует 
специальности «Вычислительные машины, комплексы, системы и сети», а 
содержание выпускной квалификационной работы -- техническому
заданию.

{\bfseries Целью} работы является улучшение 
качества обслуживания пользователей сервиса-агрегатора за счет снижения 
среднего времени ответа.

Для достижения данной цели были поставлены следующие \textbf{задачи}.
\begin{enumerate}
	\item Выполнить сравнительный анализ методов прогнозирования.
	\item Разработать алгоритм распределения нестационарной нагрузки и его 
		программную реализацию на основе выбранного метода 
		прогнозирования.
	\item Выполнить сравнение среднего времени ответа в исходной и 
		использующей разработанный алгоритм системах.
\end{enumerate}

{\bfseries Теоретическую основу исследования} составляют научные труды
отечественных и зарубежных авторов в области компьютерных технологий,
прогнозирования временных рядов, экономики, теории вероятностей и
математической статистики.

{\bfseries Методологическую основу исследования} составляет эксперимент.

{\bfseries Научная новизна работы} заключается в следующем.
\begin{enumerate}
   \item Впервые структурированы и представлены рекомендации по выбору метода
      прогнозирования нагрузки для применения в области сервисов-агрегаторов.
   \item Разработанный алгоритм распределения нестационарной нагрузки,
      в отличие от существующих, основывается на построении модели нагрузки
      и сочетает использование кеша и мгновенное перенаправление запроса к
      сторонним системам.
\end{enumerate}

{\bfseries Практическая значимость} работы заключается в том, что 
разработанный алгоритм распределения нестационарной нагрузки будет
интегрирован в разрабатываемый сервис"=агрегатор. Кроме того, 
сформулированные рекомендации могут быть использованы для осуществления 
выбора метода прогнозирования при разработке собственного алгоритма 
распределения запросов пользователей.

{\bfseries Апробация результатов исследования.} Сформулированные 
рекомендации по выбору метода прогнозирования обсуждались на VII Конгрессе 
молодых учёных, а их сравнительный анализ был представлен на XLVII научной 
и учебно-методической конференции.

{\bfseries Объем и структура работы.} 
Выпускная квалификационная работа содержит 40 страниц машинописного 
текста, восемь рисунков, семь таблиц, девять формул и список литературы, 
включающий 25 источников. Структурно работа состоит из введения, трёх 
разделов и заключения. Во введении обоснована актуальность выбранной темы, 
а также новизна исследования. Определены цели и задачи, объект и предмет 
исследования. Представлена апробация результатов исследования. Первый 
раздел содержит обзор предметной области, рассматриваются широко 
используемые методы прогнозирования временных рядов. Проводится 
сравнительный анализ методов и формулируются рекомендации по выбору 
моделей прогнозирования. Второй раздел посвящён процессу разработки 
алгоритма распределения нестационарной нагрузки, в нём описывается 
составление требований для нового алгоритма, а также проблемы его 
разработки. Приведено описание разработанных алгоритмов аккумуляции 
обучающей выборки и распределения запросов пользователей. Третий раздел 
посвящен тестированию разработанного комплекса. Приводятся результаты 
тестирования под двумя типами нагрузок. В заключении сформулированы 
основные результаты работы.
